\documentclass[a4paper]{article} 
\input{style/head.tex}

\usepackage{listings}
\usepackage{multirow}

\newcommand{\yourname}{Schweizerschule Mexiko, CDMX} % replace YOURNAME with your name
\newcommand{\yournetid}{M.Sc. William Bombardelli} % replace YOURNETID with your NetID
\newcommand{\youremail}{wbombardellis@gmail.com} % replace YOUREMAIL with your email
\newcommand{\assignmentnumber}{8} % replace X with assignment number
\newcommand{\coursename}{Java - While} % replace X with assignment number

\begin{document}
	
	\fancyhead[C]{}
\hrule \medskip
\begin{minipage}{0.295\textwidth} 
\raggedright
\footnotesize
\yourname \hfill\\ 
\yournetid \hfill\\ 
\youremail
\end{minipage}
\begin{minipage}{0.3\textwidth} 
\centering 
\large 
Class \assignmentnumber\\ 
\normalsize 
\coursename\\ 
\end{minipage}
\begin{minipage}{0.395\textwidth} 
\raggedleft
\includegraphics[]{logo}\hfill\\
\end{minipage}
\medskip\hrule 
\bigskip

	
	\section{Exercises}
	\begin{enumerate}
		\item Read the following solutions for the Fibonacci problem.
		\begin{lstlisting}[language=Java, basicstyle=\scriptsize]
public static void main(String[] args) {
	int prev;
	int num = 0;
	int next = 1;
	
	while (num <= 1000) {
		System.out.println(num);
		
		prev = num;
		num = next;
		next = prev + num;
	}
		
	System.out.println("=====================");
	////////// Alternative solution \\\\\\\\\\\\\
	
	int previousPrevious = 0;
	int previous = 1;
	int n = previousPrevious + previous;
		
	System.out.println(previousPrevious);
	System.out.println(previous);
		
	while(n <= 1000) {
		System.out.println(n);
		
		previousPrevious = previous;
		previous = n;
		n = previousPrevious + previous;
	}
}
		\end{lstlisting}
		Now complete the following table with the values of the variables $prev$, $num$ and $next$ at each iteration.
		\begin{table}[h!]
			\centering
			\begin{tabular}{| c | c | c | c |}
				\hline
				Iteration & $prev$ & $num$ & $next$ \\
				\hline
				1 & & & \\
				2 & & & \\
				3 & & & \\
				4 & & & \\
				5 & & & \\
				6 & & & \\
				7 & & & \\
				\hline
			\end{tabular}
		\end{table}
		And, finally, finish your own solution, if you have not done that yet.
		\item Write a program that reads the grades of ten students and print at the end the greatest note, the least note, the average note (arithmetic mean) and how many students passed ($grade \ge 7$).
		\item Write a program that tells whether a number is prime or not. A prime number is a natural number greater than 1 that is only divisible by 1 and by itself.
	\end{enumerate}
	
	\section{References}
	\begin{itemize}
		\item W3C Tutorial: 
		\begin{itemize}
			\item \url{https://www.w3schools.com/java/java_while_loop.asp}
		\end{itemize}
		\item Extra Exercises: \url{https://www.w3schools.com/java/exercise.asp}
		\begin{itemize}
			\item Java Loops
		\end{itemize}
		\item Solutions: \url{https://github.com/wbombardellis/java-unterricht/tree/master/Programme/08}
	\end{itemize}
\end{document}
