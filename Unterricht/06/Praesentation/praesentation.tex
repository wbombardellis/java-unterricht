\documentclass{beamer}

\usepackage{default}
\usepackage[german]{babel}
\usepackage[utf8]{inputenc}                   % replace by the encoding you are using

\usetheme{Berlin}

\newcommand{\cindent}{\hskip20pt}

%Header Settings
%\setbeamertemplate{headline}{}

%Footer Settings
\setbeamertemplate{navigation symbols}{
	\usebeamerfont{footline}%
	\usebeamercolor[fg]{footline}%
	\hspace{1em}%
	\insertframenumber/\inserttotalframenumber
}

\title[Java]{Java - If-Else}
\author[W. Bombardelli]{William Bombardelli}
\institute[Schweizerschule Mexiko]
{
	\vskip 12pt
	Schweizerschule Mexiko, Ciudad de México, Mexico \\
	\texttt{\url{https://github.com/wbombardellis/java-unterricht}}
}
\date{30 October 2019}

\makeatletter
\hypersetup{
	pdftitle = {\@title}, pdfkeywords = {Java}, pdfauthor = {\@author}
} 
\makeatother

\begin{document}
	\begin{frame}
		\titlepage
	\end{frame}
	
	\begin{frame}
		\frametitle{Organization}
		\tableofcontents
	\end{frame}

	%-------------------
	% If-Else
	%-------------------
	\section{If-Else}
	\begin{frame}
		\frametitle{If}
		\begin{itemize}
			\item Write a program that reads a double value and prints if it is positive.
		\end{itemize}
		\pause
		$if\ (a > 0)\ \{$\\
		\cindent $System.out.println(a\ + $ `` is positive." $);$\\
		$\}$\\
	\end{frame}

	\begin{frame}
		\frametitle{If-Else}
		\begin{itemize}
			\item Write a program that reads a double value and tells whether it is positive or not.
		\end{itemize}
		$if\ (a > 0)\ \{$\\
		\cindent $System.out.println(a\ + $ `` is positive." $);$\\
		$\}\ else\ \{$\\
		\cindent $System.out.println(a\ + $ `` is not positive." $);$\\
		$\}$\\
	\end{frame}

	\begin{frame}
		\frametitle{If-ElseIf-Else}
		\begin{itemize}
			\item Write a program that reads a double value and tells whether it is positive, negative or zero.
		\end{itemize}
		$if\ (a > 0)\ \{$\\
		\cindent $System.out.println(a\ + $ `` is positive." $);$\\
		$\}\ else\ if\ (a < 0) \ \{$\\
		\cindent $System.out.println(a\ + $ `` is negative." $);$\\
		$\}\ else\ \{$\\
		\cindent $System.out.println(a\ + $ `` is zero." $);$\\
		$\}$\\
	\end{frame}

	\begin{frame}
		\frametitle{Exercises}
		\begin{enumerate}
			\item Write a program that greets the user correctly according to the current time. The rule is the following:
			\begin{itemize}
				\item If current time is before 12a.m., then greet him/her with ``good morning".
				\item If current time is between 12a.m. and 16p.m., then greet him/her with ``good afternoon".
				\item If current time is after 16p.m, then greet him/her with ``good evening".
			\end{itemize}
		\end{enumerate}
	\end{frame}

	\begin{frame}
		\frametitle{Exercises}
		\begin{enumerate}[2]
			\item Extend your assignment's program further using If or If-Else.
		\end{enumerate}
	\end{frame}

	\begin{frame}
		\frametitle{If-ElseIf-Else Grammar Rules}
		$if\ (\langle \text{Boolean condition} \rangle)\ \{$\\
		\cindent $...$\\
		$\}\ [else\ if\ (\langle \text{Boolean condition} \rangle) \ \{$\\
		\cindent $...$\\
		$\}]\ [else\ \{$\\
		\cindent $...$\\
		$\}]$\\
	\end{frame}

	\begin{frame}
		\frametitle{Exercises}
		\begin{enumerate}
			\item Implement the following algorithm for finding real solutions of a quadratic polynomial $ax^2 + bx + c$ in Java.
			\begin{itemize}
				\item Read $a$, $b$ and $c$.
				\item Calculate $\Delta := b^2 - 4ac$
				\item If $\Delta \ge 0$ then print solutions $x_1$ and $x_2$
				\begin{itemize}
					\item $x_1 = \frac{-b + \sqrt{\Delta}}{2a}, x_2 = \frac{-b - \sqrt{\Delta}}{2a}$
				\end{itemize}
				\item Otherwise print that there is no real solutions for this polynomial.
			\end{itemize}
		\end{enumerate}
	\end{frame}

	\begin{frame}
		\frametitle{Exercises}
		\begin{enumerate}[2]
			\item Write a program that gets a positive integer number as input and writes whether that number is even or odd.
			\begin{itemize}
				\item Tip 1: An number $n \in \mathbb{N}$ is even if, and only if, $n$ divided by 2 gives remainder 0.
				\item Tip 2: In Java $a\ \%\ b$ results in the remainder of the whole division of $a$ by $b$.
				\item Also: Ensure that the input is positive. If it is not, then reject user's input.
			\end{itemize}
		\end{enumerate}
	\end{frame}

	\begin{frame}
	\frametitle{Exercises}
	\begin{enumerate}[3]
		\item Write a program that reads two numbers and a character representing the operation to perform over those numbers, and print the result accordingly. Example for an execution:\\
		\vskip20pt
		\texttt{Please provide the first number:}\\
		\texttt{2}\\
		\texttt{Please provide the operation: }\\
		\texttt{+}\\
		\texttt{Please provide the second number: }\\
		\texttt{3}\\
		\texttt{The result of 2 + 3 is 5}\\
	\end{enumerate}
	\end{frame}
		
	%-------------------
	% Summary
	%-------------------
	\section{Summary}
	
	\begin{frame}
		\frametitle{Summary}
		\begin{itemize}
			\item IF-Else allows you to control the flow of your program.
			\item Next Lesson: Boolean Algebra
		\end{itemize}
	\end{frame}

	\begin{frame}
		\frametitle{References}
		\begin{itemize}
			\item W3C Tutorial: 
			\begin{itemize}
				\item \url{https://www.w3schools.com/java/java_conditions.asp}
			\end{itemize}
			\item Exercises: \url{https://www.w3schools.com/java/exercise.asp}
			\begin{itemize}
				\item Java If...Else
			\end{itemize}
		\end{itemize}
		
	\end{frame}

\end{document}
