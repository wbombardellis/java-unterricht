\documentclass{beamer}

\usepackage{default}
\usepackage[german]{babel}
\usepackage[utf8]{inputenc}                   % replace by the encoding you are using

\usetheme{Berlin}

\newcommand{\cindent}{\hskip20pt}

%Header Settings
%\setbeamertemplate{headline}{}

%Footer Settings
\setbeamertemplate{navigation symbols}{
	\usebeamerfont{footline}%
	\usebeamercolor[fg]{footline}%
	\hspace{1em}%
	\insertframenumber/\inserttotalframenumber
}

\title[Java]{Java - Arrays}
\author[W. Bombardelli]{William Bombardelli}
\institute[Schweizerschule Mexiko]
{
	\vskip 12pt
	Schweizerschule Mexiko, Ciudad de México, Mexico \\
	\texttt{\url{https://github.com/wbombardellis/java-unterricht}}
}
\date{19 February 2020}

\makeatletter
\hypersetup{
	pdftitle = {\@title}, pdfkeywords = {Java}, pdfauthor = {\@author}
} 
\makeatother

\begin{document}
	\begin{frame}
		\titlepage
	\end{frame}
	
	\begin{frame}
		\frametitle{Organization}
		\tableofcontents
	\end{frame}

	%-------------------
	% Arrays
	%-------------------
	\section{Arrays}
	\begin{frame}
		\frametitle{Array}
		Arrays allow you to store several values in one variable
		\pause
		\begin{itemize}
			\item Write a program that reads the grades of ten students and print at the end the average (arithmetic mean) grade and how many students passed ($grade \ge 7$).
		\end{itemize}
		\pause
		$double[]\ grades\ =\ new\ double[10];$\\
		$for\ (int\ i = 0; i < 10; i++)\ \{$\\
			\cindent $grades[i] = reader.nextDouble();$\\
		$\}$\\
	\end{frame}

	\begin{frame}
		\frametitle{Exercises}
		\begin{enumerate}
			\item ...
		\end{enumerate}
	\end{frame}

	\begin{frame}
		\frametitle{Array Grammar Rules}
		\textbf{Array Declaration:}\\
		$\langle Type \rangle []\ \langle Varname \rangle\ =\ new\ \langle Type \rangle [SIZE];$\\
		\vskip20pt
		\textbf{Array Assignment Statement:}\\
		$\langle Varname \rangle [POSITION]\ =\ \langle \text{Expression}\rangle;$\\
		Where $POSITION \in [0, SIZE-1]$.
	\end{frame}

	\begin{frame}
		\frametitle{Exercises}
		\begin{enumerate}
			\item Write a program that reads a word from the keyboard and prints it reversed. Example: Notebook $\to$ koobetoN
			\item Write a program that tells whether a word is a palindrome or not. A palindrome is a word that reads the same backward as forward. Example: level, radar.
		\end{enumerate}
	\end{frame}

	%-------------------
	% Summary
	%-------------------
	\section{Summary}
	
	\begin{frame}
		\frametitle{Summary}
		\begin{itemize}
			\item Arrays allow you to store several values in one single variable.
			\item Next Lesson: Methods
		\end{itemize}
	\end{frame}

	\begin{frame}
		\frametitle{References}
		\begin{itemize}
			\item W3C Tutorial: 
			\begin{itemize}
				\item \url{https://www.w3schools.com/java/java_arrays.asp}
			\end{itemize}
			\item Exercises: \url{https://www.w3schools.com/java/exercise.asp}
			\begin{itemize}
				\item Java Arrays
			\end{itemize}
		\end{itemize}
		
	\end{frame}

\end{document}
