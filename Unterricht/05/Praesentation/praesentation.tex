\documentclass{beamer}

\usepackage{default}
\usepackage[german]{babel}
\usepackage[utf8]{inputenc}                   % replace by the encoding you are using

\usetheme{Berlin}

%Header Settings
%\setbeamertemplate{headline}{}

%Footer Settings
\setbeamertemplate{navigation symbols}{
	\usebeamerfont{footline}%
	\usebeamercolor[fg]{footline}%
	\hspace{1em}%
	\insertframenumber/\inserttotalframenumber
}

\title[Java]{Java - User Interface}
\author[W. Bombardelli]{William Bombardelli}
\institute[Schweizerschule Mexiko]
{
	\vskip 12pt
	Schweizerschule Mexiko, Ciudad de México, Mexico \\
	\texttt{\url{https://github.com/wbombardellis/java-unterricht}}
}
\date{16 October 2019}

\makeatletter
\hypersetup{
	pdftitle = {\@title}, pdfkeywords = {Java}, pdfauthor = {\@author}
} 
\makeatother

\begin{document}
	\begin{frame}
		\titlepage
	\end{frame}
	
	\begin{frame}
		\frametitle{Organization}
		\tableofcontents
	\end{frame}

	%-------------------
	% Review
	%-------------------
	\begin{frame}
		\frametitle{Variables Review}
		\url{https://www.youtube.com/watch?v=l26oaHV7D40} (Until 1:56)
	\end{frame}

	%-------------------
	% User interface
	%-------------------
	\section{User Interface}
	\begin{frame}
		\frametitle{Alternative User Interface}
		\begin{itemize}
			\item Not only terminal (text console)
			\item Also:
			\begin{itemize}
				\item File System
				\item Network
				\item Graphical User Interface
				\item Loudspeaker
				\item ...
			\end{itemize}
		\end{itemize}
	\end{frame}

	\begin{frame}
		\frametitle{Exercises}
		\begin{itemize}
			\item Write a program that plays the natural major scale.
			\pause
			\item Write a program that composes a melody of 10 notes using the chromatic scale and reproduces it in 60bpm.
			\pause
			\item Write a program that composes a melody of \textbf{20} notes using the \textbf{whole tone scale} and reproduces it in \textbf{120bpm}.
			\begin{itemize}
				\item The whole tone scale is composed of those notes reached with whole steps starting from the fundamental pitch. That is, starting from, say it, C, you get C, D, E, F\#, G\#, and A\#.
			\end{itemize}
		\end{itemize}
	\end{frame}

	\begin{frame}
		\frametitle{Exercises}
		\begin{itemize}
			\item Write a program that makes a 2D drawing of the models of the Hydrogen and Helium atoms at the screen.
		\end{itemize}
	\end{frame}
	
	%-------------------
	% Summary
	%-------------------
	\section{Summary}
	
	\begin{frame}
		\frametitle{Summary}
		\begin{itemize}
			\item Many possibilities for human-machine interaction.
			\item Next Week: If-Else
		\end{itemize}
	\end{frame}

	\begin{frame}
		\frametitle{References}
		\begin{itemize}
			\item \url{https://docs.oracle.com/javase/tutorial/sound}
			\item \url{https://javatutorial.net/display-text-and-graphics-java-jframe}
			\item \url{https://docs.oracle.com/javase/7/docs/api/java/awt/Graphics.html} 
		\end{itemize}
		
	\end{frame}

\end{document}
