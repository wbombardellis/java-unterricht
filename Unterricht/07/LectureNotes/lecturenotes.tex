\documentclass[a4paper]{article} 
\input{style/head.tex}

\usepackage{multirow}

\newcommand{\yourname}{Schweizerschule Mexiko, CDMX} % replace YOURNAME with your name
\newcommand{\yournetid}{M.Sc. William Bombardelli} % replace YOURNETID with your NetID
\newcommand{\youremail}{wbombardellis@gmail.com} % replace YOUREMAIL with your email
\newcommand{\assignmentnumber}{7} % replace X with assignment number
\newcommand{\coursename}{Java - Boolean Algebra} % replace X with assignment number

\begin{document}
	
	\fancyhead[C]{}
\hrule \medskip
\begin{minipage}{0.295\textwidth} 
\raggedright
\footnotesize
\yourname \hfill\\ 
\yournetid \hfill\\ 
\youremail
\end{minipage}
\begin{minipage}{0.3\textwidth} 
\centering 
\large 
Class \assignmentnumber\\ 
\normalsize 
\coursename\\ 
\end{minipage}
\begin{minipage}{0.395\textwidth} 
\raggedleft
\includegraphics[]{logo}\hfill\\
\end{minipage}
\medskip\hrule 
\bigskip

	
	\section{Boolean Algebra}
	In the classical algebra, we deal with sets of numbers and operations over them. Example for sets:
	\begin{itemize}
		\item $\mathbb{N} = \{1, 2, 3, ...\}$
		\item $\mathbb{Z} = \{..., -3, -2, -1, 0, 1, 2, 3, ...\}$
		\item $\mathbb{R} = \{..., -1.1, ..., -1, ..., 0, ..., 1, ..., 1.1, ...\}$
	\end{itemize}

	In Java, we can work with these numbers using integer or double variables.\\
	$int\ a\ =\ 1;$\\
	$int\ b\ =\ 2 + 3;$
	
	The operations for this algebra include:
	\begin{itemize}
		\item +, -, $\times$, $\div$, ...
	\end{itemize}
	In Java, we can operate over numbers using arithmetic operators.\\
	$int\ c\ =\ a + b;$\\
	$int\ d\ =\ c * b;$\\
	
	\par
	In the Boolean algebra, we work with a new set 
	\begin{equation*}
		\mathbb{B} = \{true, false\}
	\end{equation*} which contains only two values: $true$ and $false$.
	In Java, we can work with these values using boolean variables.\\
	$boolean\ a\ =\ true;$\\
	$boolean\ b\ =\ 1 > 0;$\\
	$boolean\ c\ =\ 1 == 1;$
	
	The operations over this set are three and are defined the following manner:
	\begin{itemize}
		\item AND (\&\&)
		\begin{itemize}
			\item $true\ \&\&\ true = true$
			\item $true\ \&\&\ false = false$
			\item $false\ \&\&\ true = false$
			\item $false\ \&\&\ false = false$
		\end{itemize}
		\item OR ($||$)
		\begin{itemize}
			\item $true\ ||\ true = true$
			\item $true\ ||\ false = true$
			\item $false\ ||\ true = true$
			\item $false\ ||\ false = false$
		\end{itemize}
		\item NOT (!)
		\begin{itemize}
			\item $!\ true = false$
			\item $!\ false = true$
		\end{itemize}
	\end{itemize}
	In Java, we can operate over booleans using boolean operators.\\
	$boolean\ d = true\ \&\&\ false;$\\
	$boolean\ e = true\ ||\ true;$\\
	$boolean\ f = d\ \&\&\ e;$\\
	$boolean\ g = d\ ||\ e;$\\

	\section{Exercises}
	\begin{enumerate}
		\item Write a program that takes a bi-dimensional coordinate $(x,y)$ and outputs which quadrant it belongs to.
	
		\item The Brazilian electoral legislation defines the following rules for the eligibility to vote. If the citizen is
		\begin{itemize}
			\item younger than 16, then they \textbf{may not} vote;
			\item between 16 (inclusive) and 18 (exclusive) years old or older than 64 years old, then they \textbf{may} vote;
			\item between 18 (inclusive) and 65 (exclusive) years old, then they \textbf{must} vote.
		\end{itemize}
		Now write a program that informs a citizen's eligibility to vote, based on their age.
		
		\item The body mass index ($bmi$) is calculated according to the following formula: $bmi = \frac{mass (Kg)}{height^2 (m)}$. Based on the result, a person is classified according to the following table:
		\begin{table}[h!]
			\centering
			\begin{tabular}{crr}
				Gender 					& $bmi$						& Class \\
				\hline\\
				\multirow{4}{*}{Male}	&	$bmi \le 20.7$			& underweight \\
										&	$20.7 < bmi \le 26.4$	& normal \\
										& 	$26.4 < bmi \le 31.1$	& overweight \\
										&	$31.1 < bmi $			& obese \\
				\hline\\
				\multirow{4}{*}{Female}	&	$bmi \le 19.1$			& underweight \\
										&	$19.1 < bmi \le 25.8$	& normal \\
										& 	$25.8 < bmi \le 32.3$	& overweight \\
										&	$32.3 < bmi $			& obese \\
				\hline\\
			\end{tabular}
		\end{table}
	\end{enumerate}

	\section{Summary}
	\begin{itemize}
		\item Truth Tables:\\
		\begin{tabular}{c | c c}
			OR ($||$) & True & False \\
			\hline
			True	& True & True \\
			False	& True & False\\
		\end{tabular}\\
		\par
		\begin{tabular}{c | c c}
			AND ($\&\&$) & True & False \\
			\hline
			True	& True & False \\
			False	& False & False\\
		\end{tabular}\\
		\par
		\begin{tabular}{c | c}
			NOT ($!$) &\\
			\hline
			True	& False \\
			False	& True \\
		\end{tabular}
		\item Next Lesson: While
	\end{itemize}
	
	\section{References}
	\begin{itemize}
		\item W3C Tutorial: 
		\begin{itemize}
			\item \url{https://www.w3schools.com/java/java_booleans.asp}
		\end{itemize}
		\item Exercises: \url{https://www.w3schools.com/java/exercise.asp}
		\begin{itemize}
			\item Java Booleans
		\end{itemize}
	\end{itemize}
\end{document}
