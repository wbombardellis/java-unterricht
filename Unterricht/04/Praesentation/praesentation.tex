\documentclass{beamer}

\usepackage{default}
\usepackage[german]{babel}
\usepackage[utf8]{inputenc}                   % replace by the encoding you are using

\usetheme{Berlin}

%Header Settings
%\setbeamertemplate{headline}{}

%Footer Settings
\setbeamertemplate{navigation symbols}{
	\usebeamerfont{footline}%
	\usebeamercolor[fg]{footline}%
	\hspace{1em}%
	\insertframenumber/\inserttotalframenumber
}

\title[Java]{Java - Data Types}
\author[W. Bombardelli]{William Bombardelli}
\institute[Schweizerschule Mexiko]
{
	\vskip 12pt
	Schweizerschule Mexiko, Ciudad de México, Mexico \\
	\texttt{\url{https://github.com/wbombardellis/java-unterricht}}
}
\date{2 October 2019}

\makeatletter
\hypersetup{
	pdftitle = {\@title}, pdfkeywords = {Java}, pdfauthor = {\@author}
} 
\makeatother

\begin{document}
	\begin{frame}
		\titlepage
	\end{frame}
	
	\begin{frame}
		\frametitle{Organization}
		\tableofcontents
	\end{frame}

	%-------------------
	% Review
	%-------------------
	\begin{frame}
		\frametitle{Programming Review}
		\url{https://www.youtube.com/watch?v=FCMxA3m_Imc} (Until 1:22)
	\end{frame}

	%-------------------
	% Data Types
	%-------------------
	\section{Data Types}
	\begin{frame}
		\frametitle{Data Types}
		\begin{itemize}
			\item A variable has a type
			\begin{itemize}
				\item int $\sim$ $\mathbb{Z}$
				\item double $\sim$ $\mathbb{R}$
				\item char $= \{`a`, `b`, \dots \}$
				\item boolean $= \{true, false\}$
				\item String $= \{``a", ``aa", \dots, ``b", ``bb", \dots \}$
				\item ...
			\end{itemize}
			\pause
			\item {\color{green} Type checking enhances code correctness}
			\pause
			\item {\color{red} Types increase coding-time}
		\end{itemize}
	\end{frame}

	\begin{frame}
		\frametitle{Java Grammar Rules}
		\textbf{Variable Declaration:}\\
		$\langle Type \rangle\ \langle Varname \rangle;$ or\\
		$\langle Type \rangle\ \langle Varname \rangle\ =\ \langle \text{Expression of type Type}\rangle;$\\
		\vskip20pt
		\textbf{Assignment Statement:}\\
		$\langle Varname \rangle\ =\ \langle \text{Expression of same type as variable Varname}\rangle;$\\
	\end{frame}

	\begin{frame}
		\frametitle{Exercises}
		\begin{itemize}
			\item Write a program in Java that reads two \textbf{double} values from the keyboard and prints their sum, difference, product and division at the screen/ console.
			\pause
			\item Write a program in Java that reads a temperature value in Celsius and gives it in Fahrenheit and in Kelvin.
			\pause
			\item Write a program in Java that reads a word in Spanish from the user and tells him/her whether it contains a diphthong.
			\begin{itemize}
				\item \textit{En español dos vocales en contacto se articulan como diptongo cuando:}
				\begin{itemize}
					\item \textit{una es cerrada (/i u/) no acentuada y la otra es abierta (/a e o/). O}
					\item \textit{ambas son cerradas, excepto si son iguales.}
				\end{itemize}
			\end{itemize}
		\end{itemize}
	\end{frame}
	
	%-------------------
	% Summary
	%-------------------
	\section{Summary}
	
	\begin{frame}
		\frametitle{Summary}
		\begin{itemize}
			\item int $\sim$ $\mathbb{Z}$
			\item double $\sim$ $\mathbb{R}$
			\item char $= \{`a`, `b`, \dots \}$
			\item boolean $= \{true, false\}$
			\item String $= \{``a", ``aa", \dots, ``b", ``bb", \dots \}$
			\item ...
		\end{itemize}
		\begin{itemize}
			\item Next Week: User Interface
		\end{itemize}
	\end{frame}

	\begin{frame}
		\frametitle{References}
		\begin{itemize}
			\item W3C Tutorial: 
			\begin{itemize}
				\item \url{https://www.w3schools.com/java/java\_data\_types.asp}
				\item \url{https://www.w3schools.com/java/java\_type\_casting.asp}
				\item \url{https://www.w3schools.com/java/java\_strings.asp}
				\item \url{https://www.w3schools.com/java/java_booleans.asp}
			\end{itemize}
			\item Exercises: \url{https://www.w3schools.com/java/exercise.asp}
			\begin{itemize}
				\item Java Data Types
				\item Java Strings
				\item Java Booleans
			\end{itemize}
		\end{itemize}
	\end{frame}

\end{document}
