\documentclass[a4paper]{article} 
\input{style/head.tex}

\newcommand{\yourname}{Schweizerschule Mexiko, CDMX}
\newcommand{\yournetid}{M.Sc. William Bombardelli}
\newcommand{\youremail}{wbombardellis@gmail.com}
\newcommand{\coursename}{Java}

\begin{document}
	\fancyhead[C]{}
\hrule \medskip
\begin{minipage}{0.295\textwidth} 
\raggedright
\footnotesize
\yourname \hfill\\ 
\yournetid \hfill\\ 
\youremail
\end{minipage}
\begin{minipage}{0.3\textwidth} 
\centering 
\large 
Class \assignmentnumber\\ 
\normalsize 
\coursename\\ 
\end{minipage}
\begin{minipage}{0.395\textwidth} 
\raggedleft
\includegraphics[]{logo}\hfill\\
\end{minipage}
\medskip\hrule 
\bigskip

	
	\section{Allgemein}
	\begin{itemize}
		\item Kursname: Java
		\item Lehrer: William Bombardelli
		\item Tempo: 2 Stunden pro Woche. Mittwoch 14.35
	\end{itemize}
	
	\section{Ziele}
	\begin{itemize}
		\item Probleme anhand eines Rechners lösen
		\item Programmen auf Java lesen bzw. verstehen und schreiben 
	\end{itemize}

	\section{Bewertungsprozess}
	\begin{itemize}
		\item 1. Viertel: Mitarbeit
		\item 2. Viertel: 1 Prüfung + 1 benotete Aufgabe
		\item 3. Viertel: 1 Prüfung + 1 benotete Aufgabe
	\end{itemize}

	\section{Bemerkungen}
	\begin{itemize}
		\item Web-Tutorials und Hausaufgaben sind empfohlen aber wahlweise.
		\item Diskussion mit Kollegen und Zusammenarbeit wird gefördert.
		\item Anwesenheit wird jede Woche kontrolliert.
	\end{itemize}

	\section{Literatur und Referenzen}
	\begin{itemize}
		\item Materialien: \url{https://github.com/wbombardellis/java-unterricht}
		\item Oracle Java Tutorial: \url{https://docs.oracle.com/javase/tutorial}
		\item W3C Java Tutorial: \url{https://www.w3schools.com/java}
	\end{itemize}

	\section{Kalender}
	Stand: \the\day/\the\month/\the\year.\\
	\includegraphics[width=\textwidth]{Kalender}
\end{document}
