\documentclass[a4paper]{article} 
\input{style/head.tex}

%-------------------------------
%	TITLE VARIABLES (identify your work!)
%-------------------------------

\newcommand{\yourname}{Schweizerschule Mexiko, CDMX} % replace YOURNAME with your name
\newcommand{\yournetid}{M.Sc. William Bombardelli} % replace YOURNETID with your NetID
\newcommand{\youremail}{wbombardellis@gmail.com} % replace YOUREMAIL with your email
\newcommand{\assignmentnumber}{1} % replace X with assignment number
\newcommand{\coursename}{Java} % replace X with assignment number

\begin{document}
	
	%-------------------------------
	%	TITLE SECTION (do not modify unless you really need to)
	%-------------------------------
	\fancyhead[C]{}
\hrule \medskip
\begin{minipage}{0.295\textwidth} 
\raggedright
\footnotesize
\yourname \hfill\\ 
\yournetid \hfill\\ 
\youremail
\end{minipage}
\begin{minipage}{0.3\textwidth} 
\centering 
\large 
Class \assignmentnumber\\ 
\normalsize 
\coursename\\ 
\end{minipage}
\begin{minipage}{0.395\textwidth} 
\raggedleft
\includegraphics[]{logo}\hfill\\
\end{minipage}
\medskip\hrule 
\bigskip

	
	\section{Task}
	Use your creativity to extend any of the exercises below (from class 5). Choose only one exercise.
	\begin{itemize}
		\item Write a program that composes a melody of 10 notes using the chromatic scale and reproduces it in 60bpm.
		\item Write a program that makes a 2D drawing of the models of the Hydrogen and Helium atoms at the screen.
	\end{itemize}
	To build a valid extension you need to write some extra lines of code, and not only change values, and you need to add some new feature to the program. Examples:
	\begin{itemize}
		\item Make the music composition more complex, using different time signatures or asking the user with desired parameters.
		\item Draw models of other atoms at the screen.
	\end{itemize}
	Tip:
	\begin{itemize}
		\item If you are going for a complex feature or plan to add several features, implement it one-by-one and ensure you have a good-working version between each step.
	\end{itemize}

	\section{Groups}
	You may work in pairs or alone.
	
	\section{Delivery}
	Until November, 8th, 2019 23:59, via e-mail wbombardellis@gmail.com. A zip file containing the Netbeans project with the source-code (.java files). The zip file's name must contain your first and last names.

	\section{Evaluation}
	The grade will be given using the following schema:
	\begin{itemize}
		\item 70\% Code correctness: Compiles and runs correctly.
		\item 20\% Creativity: Added features
		\item 10\% Presentation: Code organization/ indentation
	\end{itemize}
	\par \par
	\begin{itemize}
		\item Plagiarism implies invalid solution.
		\item Late delivery implies that you get only 70\% of the grade.
	\end{itemize}
\end{document}
